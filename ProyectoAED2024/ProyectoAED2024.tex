%  LaTeX support: latex@mdpi.com
%  For support, please attach all files needed for compiling as well as the log file, and specify your operating system, LaTeX version, and LaTeX editor.

%=================================================================
% pandoc conditionals added to preserve backwards compatibility with previous versions of rticles

\documentclass[notspecified,article,submit,moreauthors,pdftex]{Definitions/mdpi}


%% Some pieces required from the pandoc template
\setlist[itemize]{leftmargin=*,labelsep=5.8mm}
\setlist[enumerate]{leftmargin=*,labelsep=4.9mm}


%--------------------
% Class Options:
%--------------------

%---------
% article
%---------
% The default type of manuscript is "article", but can be replaced by:
% abstract, addendum, article, book, bookreview, briefreport, casereport, comment, commentary, communication, conferenceproceedings, correction, conferencereport, entry, expressionofconcern, extendedabstract, datadescriptor, editorial, essay, erratum, hypothesis, interestingimage, obituary, opinion, projectreport, reply, retraction, review, perspective, protocol, shortnote, studyprotocol, systematicreview, supfile, technicalnote, viewpoint, guidelines, registeredreport, tutorial
% supfile = supplementary materials

%----------
% submit
%----------
% The class option "submit" will be changed to "accept" by the Editorial Office when the paper is accepted. This will only make changes to the frontpage (e.g., the logo of the journal will get visible), the headings, and the copyright information. Also, line numbering will be removed. Journal info and pagination for accepted papers will also be assigned by the Editorial Office.

%------------------
% moreauthors
%------------------
% If there is only one author the class option oneauthor should be used. Otherwise use the class option moreauthors.

%---------
% pdftex
%---------
% The option pdftex is for use with pdfLaTeX. Remove "pdftex" for (1) compiling with LaTeX & dvi2pdf (if eps figures are used) or for (2) compiling with XeLaTeX.

%=================================================================
% MDPI internal commands - do not modify
\firstpage{1}
\makeatletter
\setcounter{page}{\@firstpage}
\makeatother
\pubvolume{1}
\issuenum{1}
\articlenumber{0}
\pubyear{2023}
\copyrightyear{2023}
%\externaleditor{Academic Editor: Firstname Lastname}
\datereceived{ }
\daterevised{ } % Comment out if no revised date
\dateaccepted{ }
\datepublished{ }
%\datecorrected{} % For corrected papers: "Corrected: XXX" date in the original paper.
%\dateretracted{} % For corrected papers: "Retracted: XXX" date in the original paper.
\hreflink{https://doi.org/} % If needed use \linebreak
%\doinum{}
%\pdfoutput=1 % Uncommented for upload to arXiv.org

%=================================================================
% Add packages and commands here. The following packages are loaded in our class file: fontenc, inputenc, calc, indentfirst, fancyhdr, graphicx, epstopdf, lastpage, ifthen, float, amsmath, amssymb, lineno, setspace, enumitem, mathpazo, booktabs, titlesec, etoolbox, tabto, xcolor, colortbl, soul, multirow, microtype, tikz, totcount, changepage, attrib, upgreek, array, tabularx, pbox, ragged2e, tocloft, marginnote, marginfix, enotez, amsthm, natbib, hyperref, cleveref, scrextend, url, geometry, newfloat, caption, draftwatermark, seqsplit
% cleveref: load \crefname definitions after \begin{document}

%=================================================================
% Please use the following mathematics environments: Theorem, Lemma, Corollary, Proposition, Characterization, Property, Problem, Example, ExamplesandDefinitions, Hypothesis, Remark, Definition, Notation, Assumption
%% For proofs, please use the proof environment (the amsthm package is loaded by the MDPI class).

%=================================================================
% Full title of the paper (Capitalized)
\Title{Análisis del mercado laboral en España}

% MDPI internal command: Title for citation in the left column
\TitleCitation{Análisis del mercado laboral en España}

% Author Orchid ID: enter ID or remove command
%\newcommand{\orcidauthorA}{0000-0000-0000-000X} % Add \orcidA{} behind the author's name
%\newcommand{\orcidauthorB}{0000-0000-0000-000X} % Add \orcidB{} behind the author's name


% Authors, for the paper (add full first names)
\Author{Catret Ruber, Pablo$^{1}$, Palazón Caballero, José
Miguel$^{1,*}$, Rosique Martínez, Marcos$^{1}$}


%\longauthorlist{yes}


% MDPI internal command: Authors, for metadata in PDF
\AuthorNames{Catret Ruber, Pablo, Palazón Caballero, José
Miguel, Rosique Martínez, Marcos}

% MDPI internal command: Authors, for citation in the left column
%\AuthorCitation{Lastname, F.; Lastname, F.; Lastname, F.}
% If this is a Chicago style journal: Lastname, Firstname, Firstname Lastname, and Firstname Lastname.
\AuthorCitation{Catret Ruber, P.; Palazón Caballero, J.M.; Rosique
Martínez, M.}

% Affiliations / Addresses (Add [1] after \address if there is only one affiliation.)
\address{%
$^{1}$ \quad Universidad de Valencia - Escola Tècnica Superior
d'Enginyeria (ETSE) Avinguda de l'Universitat, 46100 Burjassot,
Valencia; \\
}

% Contact information of the corresponding author
\corres{Correspondence: \href{mailto:jomipaca@alumni.uv.es}{\nolinkurl{jomipaca@alumni.uv.es}}.}

% Current address and/or shared authorship








% The commands \thirdnote{} till \eighthnote{} are available for further notes

% Simple summary
\simplesumm{A Simple summary goes here.}

%\conference{} % An extended version of a conference paper

% Abstract (Do not insert blank lines, i.e. \\)
\abstract{A single paragraph of about 200 words maximum. For research
articles, abstracts should give a pertinent overview of the work. We
strongly encourage authors to use the following style of structured
abstracts, but without headings: 1) Background: Place the question
addressed in a broad context and highlight the purpose of the study; 2)
Methods: Describe briefly the main methods or treatments applied; 3)
Results: Summarize the article's main findings; and 4) Conclusion:
Indicate the main conclusions or interpretations. The abstract should be
an objective representation of the article, it must not contain results
which are not presented and substantiated in the main text and should
not exaggerate the main conclusions.}


% Keywords
\keyword{keyword 1; keyword 2; keyword 3 (list three to ten pertinent
keywords specific to the article, yet reasonably common within the
subject discipline.).}

% The fields PACS, MSC, and JEL may be left empty or commented out if not applicable
%\PACS{J0101}
%\MSC{}
%\JEL{}

%%%%%%%%%%%%%%%%%%%%%%%%%%%%%%%%%%%%%%%%%%
% Only for the journal Diversity
%\LSID{\url{http://}}

%%%%%%%%%%%%%%%%%%%%%%%%%%%%%%%%%%%%%%%%%%
% Only for the journal Applied Sciences

%%%%%%%%%%%%%%%%%%%%%%%%%%%%%%%%%%%%%%%%%%

%%%%%%%%%%%%%%%%%%%%%%%%%%%%%%%%%%%%%%%%%%
% Only for the journal Data



%%%%%%%%%%%%%%%%%%%%%%%%%%%%%%%%%%%%%%%%%%
% Only for the journal Toxins


%%%%%%%%%%%%%%%%%%%%%%%%%%%%%%%%%%%%%%%%%%
% Only for the journal Encyclopedia


%%%%%%%%%%%%%%%%%%%%%%%%%%%%%%%%%%%%%%%%%%
% Only for the journal Advances in Respiratory Medicine
%\addhighlights{yes}
%\renewcommand{\addhighlights}{%

%\noindent This is an obligatory section in “Advances in Respiratory Medicine”, whose goal is to increase the discoverability and readability of the article via search engines and other scholars. Highlights should not be a copy of the abstract, but a simple text allowing the reader to quickly and simplified find out what the article is about and what can be cited from it. Each of these parts should be devoted up to 2~bullet points.\vspace{3pt}\\
%\textbf{What are the main findings?}
% \begin{itemize}[labelsep=2.5mm,topsep=-3pt]
% \item First bullet.
% \item Second bullet.
% \end{itemize}\vspace{3pt}
%\textbf{What is the implication of the main finding?}
% \begin{itemize}[labelsep=2.5mm,topsep=-3pt]
% \item First bullet.
% \item Second bullet.
% \end{itemize}
%}


%%%%%%%%%%%%%%%%%%%%%%%%%%%%%%%%%%%%%%%%%%


% tightlist command for lists without linebreak
\providecommand{\tightlist}{%
  \setlength{\itemsep}{0pt}\setlength{\parskip}{0pt}}



\usepackage{longtable}

\begin{document}



%%%%%%%%%%%%%%%%%%%%%%%%%%%%%%%%%%%%%%%%%%

\section{Introduction}\label{introduction}

The introduction should briefly place the study in a broad context and
highlight why it is important. It should define the purpose of the work
and its significance. The current state of the research field should be
reviewed carefully and key publications cited. Please highlight
controversial and diverging hypotheses when necessary. Finally, briefly
mention the main aim of the work and highlight the principal
conclusions. As far as possible, please keep the introduction
comprehensible to scientists outside your particular field of research.
Citing a journal paper
\citep{bertrand-krajewski_distribution_1998, leutnant_stormwater_2016}.
And now citing a book reference \citet{gujer_systems_2008}. Some MDPI
journals use Chicago and others use APA, this template should choose the
correct citation format for you once you specify the journal in the YAML
header.

To use endnotes, change \texttt{endnotes:\ true} in the YAML header,
then use \texttt{\textbackslash{}endnote\{This\ is\ an\ endnote.\}}.

\section{Materials and Methods}\label{materials-and-methods}

Materials and Methods should be described with sufficient details to
allow others to replicate and build on published results. Please note
that publication of your manuscript implicates that you must make all
materials, data, computer code, and protocols associated with the
publication available to readers. Please disclose at the submission
stage any restrictions on the availability of materials or information.
New methods and protocols should be described in detail while
well-established methods can be briefly described and appropriately
cited.

Research manuscripts reporting large datasets that are deposited in a
publicly available database should specify where the data have been
deposited and provide the relevant accession numbers. If the accession
numbers have not yet been obtained at the time of submission, please
state that they will be provided during review. They must be provided
prior to publication.

Interventionary studies involving animals or humans, and other studies
require ethical approval must list the authority that provided approval
and the corresponding ethical approval code.

\section{Results}\label{results}

This section may be divided by subheadings. It should provide a concise
and precise description of the experimental results, their
interpretation as well as the experimental conclusions that can be
drawn.

\subsection{Subsection Heading Here}\label{subsection-heading-here}

Subsection text here.

\subsubsection{Subsubsection Heading
Here}\label{subsubsection-heading-here}

Bulleted lists look like this:

\begin{itemize}
\tightlist
\item
  First bullet
\item
  Second bullet
\item
  Third bullet
\end{itemize}

Numbered lists can be added as follows:

\begin{enumerate}
\def\labelenumi{\arabic{enumi}.}
\tightlist
\item
  First item
\item
  Second item
\item
  Third item
\end{enumerate}

The text continues here.

\subsection{Figures, Tables and
Schemes}\label{figures-tables-and-schemes}

All figures and tables should be cited in the main text as Figure
\ref{fig:fig1}, \ref{tab:tab1}, etc. To get cross-reference to figure
generated by R chunks include the
\texttt{\textbackslash{}\textbackslash{}label\{\}} tag in the
\texttt{fig.cap} attribute of the R chunk:
\texttt{fig.cap\ =\ "Fancy\ Caption\textbackslash{}\textbackslash{}label\{fig:plot\}"}.

\begin{figure}
\centering
\includegraphics{ProyectoAED2024_files/figure-latex/fig1-1.pdf}
\caption{A figure added with a code chunk.\label{fig:fig1}}
\end{figure}

When making tables using \texttt{kable}, it is suggested to use the
\texttt{format="latex"} and \texttt{tabl.envir="table"} arguments to
ensure table numbering and compatibility with the mdpi document class.

\begin{table}[H]

\caption{\label{tab:tab1}This is a table caption. Tables should be placed in the 
             main text near to the first time they are~cited.}
\begin{tabular}[t]{lccc}
\toprule
  & mpg & cyl & disp\\
\midrule
Mazda RX4 & 21.0 & 6 & 160\\
Mazda RX4 Wag & 21.0 & 6 & 160\\
Datsun 710 & 22.8 & 4 & 108\\
Hornet 4 Drive & 21.4 & 6 & 258\\
Hornet Sportabout & 18.7 & 8 & 360\\
\bottomrule
\end{tabular}
\end{table}

For a very wide table, landscape layouts are allowed.

\startlandscape

\begin{table}[H]

\caption{\label{tab:tab2}This is a very wide table}
\begin{tabular}[t]{cccc}
\toprule
Title.1 & Title.2 & Title.3 & Title.4\\
\midrule
Entry 1 & Data & Data & This cell has some longer content that runs over
                               two lines\\
Entry 2 & Data & Data & Data\\
\bottomrule
\end{tabular}
\end{table}

\finishlandscape

\subsection{Formatting of Mathematical
Components}\label{formatting-of-mathematical-components}

This is an example of an equation:

\[
a = 1.
\]

If you want numbered equations use Latex and wrap in the equation
environment:

\begin{equation}
a = 1,
\end{equation}

the text following an equation need not be a new paragraph. Please
punctuate equations as regular text.

This is the example 2 of equation:

\begin{adjustwidth}{-\extralength}{0cm}
\begin{equation}
a = b + c + d + e + f + g + h + i + j + k + l + m + n + o + p + q + r + s + t + 
u + v + w + x + y + z
\end{equation}
\end{adjustwidth}

Theorem-type environments (including propositions, lemmas, corollaries
etc.) can be formatted as follows:

Example of a theorem:

\begin{Theorem}
Example text of a theorem

\end{Theorem}

The text continues here. Proofs must be formatted as follows:

Example of a proof:

\begin{proof}[Proof of Theorem1]
Text of the proof. Note that the phrase ``of Theorem 1'\,' is optional
if it is clear which theorem is being referred to.

\end{proof}

The text continues here.

\section{Discussion}\label{discussion}

Authors should discuss the results and how they can be interpreted in
perspective of previous studies and of the working hypotheses. The
findings and their implications should be discussed in the broadest
context possible. Future research directions may also be highlighted.

\section{Conclusion}\label{conclusion}

This section is not mandatory, but can be added to the manuscript if the
discussion is unusually long or complex.

\section{Patents}\label{patents}

This section is not mandatory, but may be added if there are patents
resulting from the work reported in this manuscript.

%%%%%%%%%%%%%%%%%%%%%%%%%%%%%%%%%%%%%%%%%%

\vspace{6pt}

%%%%%%%%%%%%%%%%%%%%%%%%%%%%%%%%%%%%%%%%%%
%% optional

% Only for the journal Methods and Protocols:
% If you wish to submit a video article, please do so with any other supplementary material.
% \supplementary{The following supporting information can be downloaded at: \linksupplementary{s1}, Figure S1: title; Table S1: title; Video S1: title. A supporting video article is available at doi: link.}

%%%%%%%%%%%%%%%%%%%%%%%%%%%%%%%%%%%%%%%%%%







%%%%%%%%%%%%%%%%%%%%%%%%%%%%%%%%%%%%%%%%%%
%% Optional

%% Only for journal Encyclopedia


%%%%%%%%%%%%%%%%%%%%%%%%%%%%%%%%%%%%%%%%%%
%% Optional
\input{"appendix.tex"}
%%%%%%%%%%%%%%%%%%%%%%%%%%%%%%%%%%%%%%%%%%
\begin{adjustwidth}{-\extralength}{0cm}

%\printendnotes[custom] % Un-comment to print a list of endnotes


\reftitle{References}
\bibliography{mybibfile.bib}

% If authors have biography, please use the format below
%\section*{Short Biography of Authors}
%\bio
%{\raisebox{-0.35cm}{\includegraphics[width=3.5cm,height=5.3cm,clip,keepaspectratio]{Definitions/author1.pdf}}}
%{\textbf{Firstname Lastname} Biography of first author}
%
%\bio
%{\raisebox{-0.35cm}{\includegraphics[width=3.5cm,height=5.3cm,clip,keepaspectratio]{Definitions/author2.jpg}}}
%{\textbf{Firstname Lastname} Biography of second author}

%%%%%%%%%%%%%%%%%%%%%%%%%%%%%%%%%%%%%%%%%%
%% for journal Sci
%\reviewreports{\\
%Reviewer 1 comments and authors’ response\\
%Reviewer 2 comments and authors’ response\\
%Reviewer 3 comments and authors’ response
%}
%%%%%%%%%%%%%%%%%%%%%%%%%%%%%%%%%%%%%%%%%%
\PublishersNote{}
\end{adjustwidth}


\end{document}
